\documentclass[11pt]{book} 
\usepackage[papersize={8.5in,11in},outer=1.0in,inner=1.0in,vmargin={1.0in,1.0in}, nofoot, bindingoffset=0.0in,headsep=0.0in]{geometry}
%
% common-preamble.tex
% Used for both main text and exercises.
%

%%% For book printing

% Lulu
%\usepackage[papersize={7.444in,9.681in},outer=1.75in,inner=0.5in,vmargin={1.5cm,1.5cm}, nofoot, bindingoffset=0.25in,headsep=0.15in]{geometry}

% CreateSpace


%papersize={18.91cm,24.589cm}

%hscale=0.65,vscale=0.85,
%\special{papersize=6in,9in}

%%!! \usepackage[T1]{fontenc} % this breaks eepicemu graphs...doesn't seem to be needed
%%\usepackage{textcomp}
%%\usepackage{lucidabr}

%\usepackage{lmodern}
%\usepackage[T1]{fontenc}

\usepackage{mathtools}

\usepackage[compact]{titlesec}
\titlespacing{\section}{0pt}{6pt}{-6pt}
\titlespacing{\subsection}{0pt}{3pt}{-6pt}
\titlespacing{\subsubsection}{0pt}{3pt}{-6pt}

%% this changes the spacing
%% \usepackage{sectsty}
%% \chapterfont{\sffamily}
%% \partfont{\sffamily}
%% \sectionfont{\sffamily}

\usepackage{mathpazo} % palatino math fonts
\usepackage{utopia} % fourier
\usepackage{helvet}
\usepackage{slatex}
\usepackage{endnotes}
\usepackage{mdwlist}
\usepackage{fncylab}
\usepackage{verbatim} 
\usepackage{enumerate}
\usepackage{xyling}
\usepackage[noauto]{chappg} %% [noauto] for continuous page numbers
\usepackage{amssymb}
%\usepackage{graphics}
\usepackage{graphicx}
\usepackage{float}
\usepackage{array}
\usepackage{mparhack} % make marginpar's work on even/odd sides correctly
\usepackage{fancyhdr}
\usepackage{color, epic, eepicemu}
\usepackage{listings} % Gives syntax highlighting for python code. 
\usepackage{textcomp} % Used for syntax highlighting. 
\usepackage{setspace}
% both packages define lofdepth
\usepackage{tocloft}
\usepackage{packages/subfigure} %\usepackage{subfigure} % my modified version
\usepackage{helvet}
\usepackage{hanging}
\usepackage[avantgarde]{quotchap}
\usepackage{makeidx}
\usepackage{lineno}

\renewcommand\chapterheadendvskip{\vspace*{\baselineskip}}
\renewcommand\chapterheadstartvskip{\vspace*{-5\baselineskip}}

\usepackage[bookmarks, pdftitle={Introduction to Computing: Explorations in Language, Logic, and Machines}, pdfauthor={David Evans}, colorlinks=true, linkcolor=black, citecolor=black, urlcolor=black, hyperindex]{hyperref}

\usepackage{url}
\urlstyle{sf}

%\def\url@smallstyle{%
 % \@ifundefined{selectfont}{\def\UrlFont{\sf}}{\def\UrlFont{\small\sffamily}}}
%\makeatother
%% Now actually use the newly defined style.
%\urlstyle{small}

%%% For per-chapter page numbering
%\makeatletter \@addtoreset{page}{chapter} \makeatother
%\pagenumbering[\thesection]{bychapter}

%!\pdfpagewidth 8.5in
%!\pdfpageheight 11in

%\pdfpagewidth 6.0in
%\pdfpageheight 9.0in

%\setlength\topmargin{0in}
%\setlength\headheight{14pt}
%\setlength\headsep{0.25in}
%\setlength\topskip{0in}
%!\setlength\textheight{8.2in}

%%%
%%% Exercises and Examples
%%%

\makeatletter
\@addtoreset{exercisenum}{chapter}
\@addtoreset{examplenum}{chapter}
\@addtoreset{explorationnum}{chapter}
\makeatother

\newcommand{\listexamplename}{{\large\bfseries List of Examples}\vspace*{-2ex}}
\newlistof{example}{exp}{\listexamplename}
\cftsetindents{example}{1.5em}{3.0em}

\newcommand{\listexplorationname}{{\large\bfseries List of Explorations}\vspace*{-2ex}}
\newlistof{exploration}{explore}{\listexplorationname}
\cftsetindents{exploration}{1.5em}{3.0em}

\newcounter{exercisenum}[chapter]
\newcounter{explorationnum}[chapter]
\newcounter{examplenum}[chapter]

\renewcommand\theexercisenum{\arabic{chapter}.\arabic{exercisenum}}
\renewcommand\theexamplenum{\arabic{chapter}.\arabic{examplenum}}
\renewcommand\theexplorationnum{\arabic{chapter}.\arabic{explorationnum}}

%%! \newenvironment{example}[1]{\refstepcounter{examplenum}\addcontentsline{exp}{example}{\protect\numberline{\theexamplenum} #1}\textbf{Example \theexamplenum: #1.}\quad}{}

\newenvironment{exercisenote}{}{}

\newenvironment{example}[1]{
   \refstepcounter{examplenum}
   \colorbox{examplecolor}{\makebox[1.0\textwidth][l]{\textcolor{white}{\textbf{Example \theexamplenum: #1\\}
   \addcontentsline{example}{example}{\protect\numberline{\theexamplenum}#1}
   \hspace\fill}}}\vspace*{0.5ex}}{\par \vspace*{-1.2ex}\colorbox{examplecolor}{\makebox[1.0\textwidth][l]{\ }}\vspace*{0.5ex}}

\newenvironment{examplenobar}[1]{
   \refstepcounter{examplenum}
   \colorbox{examplecolor}{\makebox[1.0\textwidth][l]{\textcolor{white}{\textbf{Example \theexamplenum: #1\\}
   \addcontentsline{example}{example}{\protect\numberline{\theexamplenum}#1}
   \hspace\fill}}}\vspace*{0.5ex}}{}

\newenvironment{exercise}{\refstepcounter{exercisenum}\textbf{Exercise \theexercisenum.\ }}{\vspace*{0.5ex}}
\newenvironment{shortexercise}{\refstepcounter{exercisenum}\textbf{Exercise \theexercisenum.\ }}{\par}

\definecolor{explorecolor}{rgb}{0.6,.25,0.15}
\definecolor{exploreendcolor}{rgb}{0.3,.05,0.05}

\definecolor{examplecolor}{rgb}{0.15,.45,0.15}

\definecolor{biocolor}{rgb}{0.15,.25,0.6}

\newenvironment{exploration}[1]{
   \refstepcounter{explorationnum}
   \colorbox{explorecolor}{\makebox[1.0\textwidth][l]{\textcolor{white}{\textbf{Exploration \theexplorationnum: #1 \\}
   \addcontentsline{explore}{exploration}{\protect\numberline{\theexplorationnum} #1}
   \hspace\fill}}}}{\vspace*{-1.2ex}\colorbox{explorecolor}{\makebox[1.0\textwidth][l]{\ }}}

\newenvironment{biography}[1]{\colorbox{biocolor}{\makebox[1.0\textwidth][l]{\textcolor{white}{\textbf{Profile: #1 \\}\hspace\fill}}}}{\vspace*{-1.2ex}\colorbox{biocolor}{\makebox[1.0\textwidth][l]{\ }}}

\newenvironment{explorationnobar}[1]{\refstepcounter{explorationnum}
   \addcontentsline{explore}{exploration}{\protect\numberline{\theexplorationnum} #1}
   \colorbox{explorecolor}{\makebox[1.0\textwidth][l]{\textcolor{white}{\textbf{Exploration \theexplorationnum: #1 \\}\hspace\fill}}}}{}

\newcommand{\shortex}[1]{\begin{shortexercise}#1\end{shortexercise}}

%\newcommand{\ex}[1]
%   {\begin{minipage}[t]{1.0\textwidth}\setlength{\parindent}{0em}\setlength{\parskip}{1.5ex}
%    \begin{exercise}#1\end{exercise}\end{minipage}}
%\newcommand{\splitex}[1]{\begin{exercise}#1\end{exercise}}

\newcommand{\beforeex}{\begin{minipage}[t]{1.0\textwidth}\setlength{\parindent}{0em}\setlength{\parskip}{1.5ex}}
\newcommand{\afterex}{\end{minipage}}
\newcommand{\beforesplitex}{}
\newcommand{\aftersplitex}{}


\newcommand{\excur}[2]
   {\begin{minipage}[t]{1.0\textwidth}\setlength{\parindent}{0em}\setlength{\parskip}{1.5ex}
    \begin{exploration}{#1}#2\end{exploration}\end{minipage}}
\newcommand{\splitexcur}[2]{\begin{exploration}{#1}#2\end{exploration}}

\newcommand{\explore}[2]
   {\begin{minipage}[t]{1.0\textwidth}\setlength{\parindent}{0em}\setlength{\parskip}{1.5ex}
    \begin{exploration}{#1}#2\\ \end{exploration}\end{minipage}}
\newcommand{\splitexplore}[2]{\begin{exploration}{#1}\vspace*{0.5ex}\par #2 \par \end{exploration}}
\newcommand{\splitexplorenobar}[2]{\begin{explorationnobar}{#1}\vspace*{0.5ex}\par #2 \par \end{explorationnobar}}

\newcommand{\splitbio}[2]{\begin{biography}{#1}\vspace*{0.5ex}\par #2 \par \end{biography}}


\newcounter{subexercisenum}[exercisenum]
\newenvironment{subexerciselist}
	{\vspace*{-2ex}\begin{enumerate}[\bfseries a.]\setlength{\parsep}{3pt}\setlength{\topsep}{3pt}\setlength{\partopsep}{0pt}\setlength{\itemsep}{0cm}} %!! \setlength{\parskip}{0cm}}
	{\end{enumerate}\vspace*{-1.3ex}}

\definecolor{gold}{rgb}{0.85,.66,0}
\definecolor{lightgrey}{rgb}{0.4,0.4,0.9}
%\newcommand{\bluestar}{\textcolor{blue}{$\left[\star\right]$}\hspace{0.5em}}
%\newcommand{\greenstar}{\textcolor{green}{$\left[\star\right]$}\hspace{0.5em}}
%\newcommand{\goldstar}{\textcolor{gold}{$\left[\star\right]$}\hspace{0.5em}}
%\newcommand{\doublegoldstar}{\textcolor{gold}{$\left[\star\star\right]$}\hspace{0.5em}}
%\newcommand{\triplegoldstar}{\textcolor{gold}{$\left[\star\star\star\right]$}\hspace{0.5em}}
%\newcommand{\quadgoldstar}  {\textcolor{gold}{$\left[\star\star\star\star\right]$}\hspace{0.5em}}

\newcommand{\bluestar}{} % \textcolor{blue}{$\left[\star\right]$}\hspace{0.5em}}
\newcommand{\greenstar}{} %\textcolor{green}{$\left[\star\right]$}\hspace{0.5em}}
\newcommand{\goldstar}{\textcolor{gold}{$\left[\star\right]$}\hspace{0.5em}}
\newcommand{\doublegoldstar}{\textcolor{gold}{$\left[\star\star\right]$}\hspace{0.5em}}
\newcommand{\triplegoldstar}{\textcolor{gold}{$\left[\star\star\star\right]$}\hspace{0.5em}}
\newcommand{\quadgoldstar}  {\textcolor{gold}{$\left[\star\star\star\star\right]$}\hspace{0.5em}}

\newcommand{\difficulty}[1]{(#1) }

\newcommand{\hint}[1]{{\footnotesize {\bold{Hint:} #1}}}

\DeclarePairedDelimiter{\ceil}{\lceil}{\rceil}

\newcommand{\solution}[1]{\par \shortsection{Solution}#1\vspace*{0.5ex}}
\newcommand{\exercisesonly}[1]{#1}


% I can't get extract to produce labels correctly without breaking things...just manually update this!
\newlabel{fig:three-bits}{{1.1}{}{}{}{}}
\newlabel{fig:subnetwork-rtn}{{2.3}{24}{Recursive transition network with subnetworks}{figure.2.3}{}}
\newlabel{example:wholenumbers}{{2.1}{28}{Replacement Grammars}{examplenum.2.1}{}}
\newlabel{fig:andsubnetwork-rtn}{{2.4}{24}{Alternate \nonterminal {Noun} subnetwork}{figure.2.4}{}}
\newlabel{example:bigger}{{3.3}{49}{Decisions\relax }{examplenum.3.3}{}}
\newlabel{ex:factorial}{{4.1}{58}{Recursive Problem Solving\relax }{examplenum.4.1}{}}
\newlabel{sec:fcompose}{{4.2.1}{55}{Procedures as Inputs and Outputs\relax }{subsection.4.2.1}{}}

\newlabel{sec:examining-lists}{{5.4.1}{85}{Procedures that Examine Lists\relax }{subsection.5.4.1}{}}
\newlabel{example:intsto}{{5.8}{92}{Procedures that Construct Lists\relax }{examplenum.5.8}{}}
\newlabel{example:list-sum}{{5.2}{86}{Procedures that Examine Lists\relax }{examplenum.5.2}{}}
\newlabel{exploration:pegboard}{{5.2}{96}{Data Abstraction\relax }{explorationnum.5.2}{}}
\newlabel{sec:listsoflists}{{5.5}{93}{Lists of Lists\relax }{section.5.5}{}}
\newlabel{sec:generic}{{5.4.2}{87}{Generic Accumulators\relax }{subsection.5.4.2}{}}
\newlabel{example:map}{{5.4}{89}{Procedures that Construct Lists\relax }{examplenum.5.4}{}}

\newlabel{fig:three-and-compose}{{6.3}{114}{Computing \function {and3} by composing two \function {and} functions}{figure.6.3}{}}

\newlabel{sec:measuringcost}{{7.1}{127}{Empirical Measurements\relax }{section.7.1}{}}
\newlabel{picture:odometer}{{7.5}{145}{Linear Growth\relax }{examplenum.7.5}{}}

\newlabel{sec:indexed-search}{{8.2.3}{171}{Indexed Search\relax }{subsection.8.2.3}{}}

\newlabel{fig:circular-pair}{{9.5}{189}{Mutable pair created by evaluating (\var {set-mcdr!} \var {pair} \var {pair})}{figure.9.5}{}}

\newlabel{fig:bb}{{12.3}{251}{Two-state Busy Beaver Machine}{figure.12.3}{}}

\setlength\headheight{14pt}
\begin{document}
%%%
%%% style.sty
%%%

%%% from http://www.f.kth.se/~ante/latex.php
\setlength{\marginparwidth}{1.0in}
%\let\oldmarginpar\marginpar
%\renewcommand\marginpar[1]{\-\oldmarginpar[\raggedleft\footnotesize #1]%
%                           {\raggedright\footnotesize #1}}

\setlength{\parindent}{0em}
\setlength{\parskip}{1.5ex}

\setlength{\floatsep}{0.5ex}
\setlength{\textfloatsep}{0.0ex}
\setlength{\abovecaptionskip}{0.25ex}
\setlength{\belowcaptionskip}{1.5ex}

%%%
%%% Colors
%%%

\definecolor{darkblue}{rgb}{0.1,0,0.55} 
\definecolor{darkgreen}{rgb}{0.1,0.55,0.1} 
\definecolor{purple}{rgb}{0.8,0.0,0.8}

%%%
%%% Quotes
%%%

\newcommand{\chapquote}[2]{\begin{flushright}\parbox{10cm}{
   \setlength{\parskip}{0.5ex}\setlength{\parindent}{0em}\footnotesize \noindent {\raggedright {\em #1}\hfill \\} 
   {\raggedleft #2 \\}}\end{flushright}}

\newcommand{\chapquoter}[2]{\begin{flushright}\parbox{10cm}{
   \setlength{\parskip}{0.5ex}\setlength{\parindent}{0em}\footnotesize \noindent {\raggedleft {\em #1}\hfill \\} 
   {\raggedleft #2 \\}}\end{flushright}}

\newcommand{\chapquotew}[3]{\begin{flushright}\parbox{#3}{
   \setlength{\parskip}{0.5ex}\setlength{\parindent}{0em}\footnotesize \noindent {\raggedleft {\em #1}\hfill \\} 
   {\raggedleft #2 \\}}\end{flushright}}

\newcommand{\chapquotewl}[3]{\parbox{#3}{\setlength{\parskip}{0.5ex}\setlength{\parindent}{0em}
   \footnotesize \noindent {\raggedright {\em #1}\hfill \\} {\raggedright #2 \\}}}

%%%
%%% Margin Quotes and Pictures
%%%

\newcommand{\marginquote}[2]{\marginpar[\footnotesize \raggedleft {\em #1} \\ 
                                        \scriptsize #2]
                            {\footnotesize \raggedright {\em #1} \\ \scriptsize #2}}
\newcommand{\sidequote}{\marginquote}

%\newcommand{\longquote}[2]{\vspace*{0.2ex}{\parbox{12cm}{\setlength{\parskip}{0.5ex}\setlength{\parindent}{0em}\noindent{\flushleft \small #1\hfill\\ \raggedleft #2\\}\vskip 1ex}}}

\newcommand{\longquote}[2]{\begin{center}\begin{minipage}[c]{0.92\textwidth}\small\setlength{\parskip}{0.5ex}\setlength{\parindent}{0em}\noindent #1 \par {\raggedleft #2\\} \end{minipage}\end{center}}

\newcommand{\margintag}[1]{\mbox{}\marginpar[\footnotesize \raggedleft \hspace{0pt} 
                           \em \textcolor{blue}{#1}]{\footnotesize \raggedright \em \textcolor{blue}{#1}}}

\newcommand{\indexeddefinition}[2]{\index{general}{#2|textbf}\emph{#1}\mbox{}\marginpar[\footnotesize \raggedleft \hspace{0pt} \em \textcolor{blue}{#1}]
                            {\footnotesize \raggedright \em \textcolor{blue}{#1}}}

%% can't have lowercase here, messes up alphabetizing?
\newcommand{\definition}[1]{\index{general}{#1|textbf}\emph{#1}\mbox{}\marginpar[\footnotesize \raggedleft \hspace{0pt} \em \textcolor{blue}{#1}]
                            {\footnotesize \raggedright \em \textcolor{blue}{#1}}}


\newcommand{\sidenote}[1]{\marginpar[\footnotesize \raggedleft #1]{\footnotesize \raggedright #1}}
\newcommand{\marginnote}{\sidenote}

\newcommand{\sidepicture}[4]{
   \marginpar{\setlength{\baselineskip}{4pt}\centering{\scalebox{#1}{\includegraphics{#2}}}\\
   \setlength{\parskip}{5pt}{\bf\footnotesize #3}\\ \tiny #4}}
\newcommand{\sidepicturenocap}[2]{
   \marginpar{\begin{center}{\scalebox{#1}{\includegraphics{#2}}}\end{center}}}

%%%\newcommand{\chintrac}{Chapter~13}

%%%
%%% Captions
%%%

% based on: http://dcwww.camd.dtu.dk/~schiotz/comp/LatexTips/LatexTips.html#captfont
% Different font in captions

\makeatletter  % Allow the use of @ in command names
\long\def\@makecaption#1#2{%
  \vskip\abovecaptionskip
  \sbox\@tempboxa{{\bf #1. #2}}%
  \ifdim \wd\@tempboxa >\hsize
    {\bf #1. #2\par}
  \else
    \hbox to\hsize{\hfil\box\@tempboxa\hfil}%
  \fi
  \vskip\belowcaptionskip}
\makeatother   % Cancel the effect of \makeatletter

\newcommand{\subcap}[1]{\vspace*{-1ex}\centering{\parbox{11cm}{\footnotesize #1}\vskip 1ex}}
\newcommand{\subcapw}[1]{\vspace*{-1ex}\centering{\parbox{12cm}{\setlength{\parskip}{1.0ex}\footnotesize #1\vskip 1ex}}}
\newcommand{\subcapc}[1]{\vspace*{-1ex}\centering{\footnotesize #1}}
\newcommand{\subcaps}[1]{\vspace*{0ex}{\parbox{11.8cm}\tiny #1\vskip 1ex}}

\newcommand{\pevalsto}{$\Rightarrow$\ }
\newcommand{\evalsto}{$\Rightarrow$\ }

\newenvironment{pythoninteracts}
  {\begin{smallquote}\begin{tabbing}\hspace*{3.5cm}\=\kill}
  {\end{tabbing}\vspace*{-0.5ex}\end{smallquote}}

\newenvironment{pythoninteractsm}[1]
  {\begin{smallquote}\begin{tabbing}\pycode|#1|\qquad \=\kill}
  {\end{tabbing}\vspace*{-0.5ex}\end{smallquote}}

\newcommand{\pyi}[2]{{$\gg$} \pycode|#1|\>\pevalsto \pycode|#2|\\}
\newcommand{\pyl}[2]{{$\gg$} \pycode|#1|\\ \ \>\pevalsto \pycode|#2|\\}
\newcommand{\pye}[2]{{$\gg$} \pycode|#1|\>\pevalsto \pbugresult{#2}\\}
\newcommand{\pyn}[1]{{$\gg$} \pycode|#1|\>\\}

%%%
%%% BNF Grammars
%%%

\newcommand{\nonterminal}[1]{{\sl #1}}
\newcommand{\terminal}[1] {{\textbf{#1}}}
\newcommand{\produces}{::$\Rightarrow$}
\newcommand{\bnfruleplain}[2]{\hspace*{1em}\nonterminal{#1}\quad\produces\quad#2 \\}
\newcommand{\bnfshowrule}[2]{\nonterminal{#1} \produces\ #2}

\newenvironment{bnfgrammar}{
   \begin{centernospace}\begin{tabbing} \hspace*{10em}\=\ \produces\ \= \kill}{\end{tabbing}\end{centernospace}}

% bnfgrammarm: argument is the longest left hand side for setting up spacing
\newenvironment{bnfgrammarm}[1]
  {\begin{smallquote}\begin{minipage}[t]{1.0\textwidth}\begin{tabbing} \nonterminal{#1}\ \=\ \produces\ \=\kill}
  {\end{tabbing}\end{minipage}\end{smallquote}\vspace*{-1ex}}

% bnfgrammarm: argument is the longest left hand side for setting up spacing
\newenvironment{bnfgrammarmc}[1]
  {\begin{centernospace}\begin{minipage}[t]{1.0\textwidth}\begin{tabbing} \nonterminal{#1}\ \=\ \produces\ \=\kill}
  {\end{tabbing}\end{minipage}\end{centernospace}\vspace*{-1ex}}

\newenvironment{bnfgrammarn}
   {\begin{centernospace}\begin{tabbing} \hspace*{3em}\=\hspace*{10em}\=\ \produces\ \= \kill}
   {\end{tabbing}\end{centernospace}}

\newenvironment{bnfgrammarmn}[1]
   {\begin{centernospace}\begin{tabbing}\hspace*{3em}\=\hspace*{3em}\=\nonterminal{#1}\ \=\ \produces\ \= \kill}
   {\end{tabbing}\end{centernospace}}

\newcommand{\bnfruleg}[2]{\centering
   \begin{tabular}{p{9.5em}p{2.5em}p{21em}}\nonterminal{#1} & \produces & \raggedright #2 \tabularnewline \end{tabular}}

\newcommand{\bnfrule}[2]{\hfill\nonterminal{#1}\>\produces\>#2\\}

\newcommand{\bnfrulen}[3]{\>\hfill\emph{#1.}\>\nonterminal{#2}\>\produces\>#3\\}

\newenvironment{widen}{\begin{minipage}[t]{1.1\textwidth}}{\end{minipage}}

\newenvironment{centernospace}{\begin{center}\vspace*{-6pt}}{\end{center}\vspace*{-6pt}}

%%%
%%% Formatting
%%%

\renewcommand{\bold}[1]{{\textbf{#1}}}
\newcommand{\symbolfont}[1]{{\sffamily #1}}


\newcommand{\tagpar}[1]{\bold{#1. }}
\newcommand{\shortsection}[1]{\textbf{#1. }}

\newcommand{\evalrule}[1]{\begin{quote}\raggedright #1\end{quote}}

\newcommand{\mevalrule}[2]{\begin{center}\vspace*{-6pt}\begin{minipage}[c]{0.9\textwidth}\setlength{\parindent}{0em}\setlength{\parskip}{1.5ex}{\bf #1.} #2\end{minipage}\end{center}}
\newcommand{\mevalrulef}[2]{\begin{center}\vspace*{-6pt}\begin{minipage}[c]{0.9\textwidth}\setlength{\parindent}{0em}\setlength{\parskip}{1.5ex}#1. #2\end{minipage}\end{center}}

%\newenvironment{code}{\begin{quote}}{\end{quote}}
\newenvironment{code}{\vspace*{-2ex}\begin{list}{}{%
  \setlength\leftmargin{1em}\setlength\labelwidth{0pt}\setlength\itemindent{0pt}}\item[]}{\end{list}\vspace*{-1.3ex}} % this is still used for formatting interactions
\newenvironment{enumtight}{\vspace*{-2ex}\begin{enumerate}\setlength{\itemsep}{0cm} \setlength{\parskip}{0cm}\setlength{\parskip}{0cm}\setlength{\parsep}{3pt}\setlength{\topsep}{3pt}\setlength{\partopsep}{0pt}}{\end{enumerate}\vspace*{-1.3ex}}
\newenvironment{itemtight}{\vspace*{-2ex}\begin{itemize}\setlength{\itemsep}{0cm} \setlength{\parskip}{0cm}\setlength{\parsep}{3pt} \setlength{\topsep}{3pt}\setlength{\partopsep}{0pt}}{\end{itemize}\vspace*{-1.3ex}}

\newenvironment{smallquote}{\vspace*{-2ex}\begin{list}{}{%
  \setlength\rightmargin{1.5em}\setlength\leftmargin{1.5em}\setlength\labelwidth{0pt}\setlength\itemindent{0pt}}\item[]}{\end{list}\vspace*{-1.3ex}} % this is still used for formatting interactions

\newenvironment{descriptionlist}
  {\begin{list}{}{\setlength\leftmargin{0.3in}
                  \setlength\labelwidth{0pt}%
                  \setlength\itemindent{-0.15in}%
                  \setlength\topsep{0pt}
                  \setlength\itemsep{0pt}
                  }}
                  {\end{list}}

\newenvironment{tightdescriptionlist}
  {\begin{list}{}{\setlength\leftmargin{0.5in}
                  \setlength\labelwidth{0pt}%
                  \setlength\itemindent{-0.15in}%
                  \setlength\topsep{0pt}
                  \setlength\itemsep{-3pt}
                  }}
                  {\end{list}}
                  
\newenvironment{procedurelist}
  {
  \begin{list}{}{ \renewcommand\makelabel[1]{##1\mbox{}\\}
  								\setlength\leftmargin{0.5in}
                  \setlength\labelwidth{0pt}%
                  \setlength\itemindent{-0.15in}%
                  \setlength\topsep{0pt}
                  \setlength\itemsep{-3pt}
                  }}
                  {\end{list}}
                  
\newcommand{\forcenl}{\ \\}
\newcommand{\function}[1]{\emph{#1}}
\newcommand{\var}[1]{\emph{#1}}
\newcommand{\sresult}[1]{{\sffamily {\small #1}}}

\newcommand{\charme}[1]{\texttt{#1}}
\newcommand{\scharme}[1]{\texttt{#1}}
\newcommand{\lzcharme}[1]{\texttt{#1}}

\newcommand{\keyword}[1]{{\sffamily #1}}

\newcommand{\pycode}{\lstinline} % active style must be python style
\newcommand{\presult}{\lstinline}
\newcommand{\pbugresult}[1]{\textcolor{red}{\sffamily #1}}
\newcommand{\bugresult}[1]{{\scalebox{0.3}{\includegraphics{images/stop-32x32.png}}} \textcolor{red}{\sffamily #1}}

\newcommand{\tmtext}[1]{{\tt #1}}
\newcommand{\bits}[1]{{\sf #1}}
\newcommand{\bit}[1]{{\sf #1}}

\newcommand{\snumber}[1]{{\sf \small #1}}
\newcommand{\soutput}[1]{{\sffamily {\small {\sl #1}}}}

\newcommand{\problemold}[3]
   {\begin{tabular}{rp{4em}p{20em}} \emph{#1} & \bold{Input:} & \raggedright #2 \tabularnewline 
                                                               & \bold{Output:} & \raggedright #3 \tabularnewline 
     \end{tabular}}
 
 
\newcommand{\problemname}[1]{{\sc\em #1}}

\newcommand{\problem}[3]
{
\begin{smallquote}
{\centering \problemname{#1}\\}
\begin{hangparas}{.25in}{1}
\bold{Input:} #2

\bold{Output:} #3
\end{hangparas}
\end{smallquote}
}
                                                                            
%%\newcommand{\url}[1]{{\sffamily {\small \em #1}}}

\newcommand{\groupcontent}[1]{\begin{minipage}[t]{1.0\textwidth}#1\end{minipage}}

%\renewenvironment{theindex}
%   {\chapter*{\indexname} %
%    \setlength\parindent{ 0pt} \setlength\parskip{ 0pt plus 0.3pt} %
%    \setlength\columnseprule{ 0pt} \setlength\columnsep{ 35pt}%
%    \let\item\@idxitem}
%   {}


%%%
%%% Page Headers and Footers
%%%

\pagestyle{fancy}

\fancyhead{}
\fancyhead[RE]{\nouppercase{\bfseries \rightmark}} 
\fancyhead[LO]{\nouppercase{\bfseries \leftmark}} 
\fancyhead[LE,RO]{\thepage}

%\fancyhead[RE]{\iffloatpage{}{\nouppercase{\bfseries \rightmark}}} 
%\fancyhead[LO]{\iffloatpage{}{\nouppercase{\bfseries \leftmark}}} 
%\fancyhead[LE,RO]{\iffloatpage{}{\thepage}}
\fancyfoot{}
\renewcommand{\headrulewidth}{0.4pt}
%\renewcommand{\headrulewidth}{\iffloatpage{0pt}{0.4pt}}
% http://www.ctan.org/tex-archive/macros/latex/contrib/fancyhdr/fancyhdr.pdf

\fancypagestyle{plain}{
  \fancyhf{} 
 % \fancyfoot[C]{\fbox{\footnotesize David Evans, \emph{Computing: Explorations in Language, Logic, and Machines}, \today}}
  \renewcommand{\headrulewidth}{0pt}
  \renewcommand{\footrulewidth}{0pt}
}

%%%
%%% Temporary Markers
%%%

\newcommand{\TODO}[1]{\hrule \textbf{TODO:} #1 \hrule }
\newcommand{\LATER}[1]{}
\newcommand{\rnote}[1]{\begin{quote}{\em #1}\end{quote}}

\newcommand{\topics}[1]{}
\newcommand{\cut}[1]{}

%%%
%%% Math Mode Text
%%% 

\newcommand{\Fibonacci}{\textrm{\emph{Fibonacci}}} % in math mode
\newcommand{\RunningTime}{\textrm{\emph{Max-Steps}}} % in math mode
\newcommand{\MaxSteps}{\textrm{\emph{Max-Steps}}_{\textrm{\textsl{Proc}}}}
\newcommand{\RunningTimeL}{\textrm{\emph{Max-Steps}}_{\textrm{\emph{list-append}}}} % in math mode
\newcommand{\TreeNodes}{\textrm{\emph{TreeNodes}}} % in math mode
\newcommand{\TreeLeaves}{\textrm{\emph{TreeLeaves}}} % in math mode

%%%
%%% Special Symbols
%%%

\newcommand{\tb}{}
\newcommand{\true}{\var{true}}
\newcommand{\false}{\var{false}}
\newcommand{\True}{\pycode|True|}
\newcommand{\False}{\pycode|False|}

\newcommand{\lparen}{(}
\newcommand{\rparen}{)}

\newcommand{\Goedel}{G\"{o}del}
\newcommand{\Godel}{G\"{o}del}
\newcommand{\GEB}{\emph{GEB}}
\newcommand{\SICP}{Abelson \& Sussman, \emph{Structure and Interpretation of Computer Programs}}

\newcommand{\bO}{{\bf \sf 0}}
\newcommand{\bI}{{\bf \sf 1}}

\newcommand{\hash}{{\tt \#}}

%%%
%%% Evaluations
%%% 

\newenvironment{aside}{\hfill\renewcommand\makeLineNumber{\rlap{\hspace{\textwidth}}}\color{lightgrey}\footnotesize}{\color{black}\par}
\newenvironment{scaside}{\hfill\renewcommand\makeLineNumber{\rlap{\hspace{\textwidth}}}\color{lightgrey}\footnotesize}{\color{black}}
\newcommand{\sidemark}[1]{\begin{aside}{\em #1}\end{aside}}
\newenvironment{saside}{\hfill \color{lightgrey}\footnotesize}{\color{black}\par}
\newcounter{evalstep}[chapter]
\newcommand{\evalstart}{\setcounter{evalstep}{0} \linenumbers[0] \setlength{\parskip}{0.1ex}}
		%	{\begin{tabbing} \hspace*{2em}\=\kill\setcounter{evalstep}{0}}
\newcommand{\sevalstep}{\stepcounter{evalstep} \linenumbers[\theevalstep]} %  \theevalstep. \>}
\newcommand{\evalstep}[2]{\sevalstep \schemesmall{#1} \par \begin{aside} #2 \end{aside}} % begin{minipage}[r]{0.90\textwidth} #2 \end{minipage}}
\newcommand{\evalstepa}[2]{\sevalstep \schemesmall{#1} \begin{saside} #2 \end{saside}} % begin{minipage}[r]{0.90\textwidth} #2 
\newcommand{\evalstepb}[2]{\sevalstep \schemesmall{#1} \par \begin{aside} #2 \end{aside}} % begin{minipage}[r]{0.90\textwidth} #2 
\newcommand{\revalstep}[2]{\sevalstep \schemesmall{#1} \par}
\newcommand{\evalsteptiny}[2]{\sevalstep \schemetiny{#1} \par \begin{aside} #2 \end{aside}}
\newcommand{\hevalstep}[2]{\stepcounter{evalstep}}
\newcommand{\hevalsteptiny}[2]{\stepcounter{evalstep}}
\newcommand{\evalindent}{\addtolength{\parindent}{0.6cm}} %\end{tabbing}\begin{tabbing}\hspace*{4em}\=\kill}
\newcommand{\evalunindent}{\addtolength{\parindent}{-0.6cm}} %\end{tabbing}\begin{tabbing}\hspace*{2em}\=\kill}
\newcommand{\evalend}{\nolinenumbers \setlength{\parskip}{1.5ex}} %\end{tabbing}}
\newcommand{\evalmark}[1]{\bold{#1}}
%\newcommand{\aside}[1]{\hspace*{5em}#1\\}
\newcommand{\schemesmall}[1]{\texttt{\small #1}}
\newcommand{\schemetiny}[1]{\texttt{\footnotesize #1}}

%%%
%%% SLaTeX
%%%

\leftcodeskip=1em % 0pt plus 1fil
\rightcodeskip=0pt plus 1fil
\abovecodeskip=0pt
\belowcodeskip=0pt

% for StaticCharme
\setkeyword{Number}  
\setkeyword{Boolean}
\setkeyword{:}

\setconstant{StaticCharme>}
\setconstant{LazyCharme>}
\setconstant{Charme>}

\setvariable{true false}  % was \setconstant
\setvariable{trace}
\setvariable{null} % make null always typeset in italics

\setvariable{and} % these are special forms, but better typeset as variables
\setvariable{or}

\def\constantfont#1{{\mbox{\small\sf#1}}}
\let\datafont\constantfont

\setspecialsymbol{\arrow}{$\rightarrow$}
\setspecialsymbol{\rest}{$\ldots$} 
\setspecialsymbol{\recproc}{\textcolor{darkblue}{\textsf{\sl Recursive-Procedure}}}
\setspecialsymbol{\basecaseres}{\textcolor{darkgreen}{\textsf{\sl Base-Case-Result}}}
\setspecialsymbol{\accumfunc}{\textcolor{darkgreen}{\textsf{\sl Accumulator-Function}}}
\setspecialsymbol{\value}{\textsf{\sl Value}}
\setspecialsymbol{\List}{\textsf{\sl List}}
\setspecialsymbol{\Expression}{\textsf{\sl Expression}}
\setspecialsymbol{\Name}{\textsf{\sl Name}}
\setspecialsymbol{\Type}{\textsf{\sl Type}}
\setspecialsymbol{\Parameters}{\textsf{\sl Parameters}}

\setspecialsymbol{\MoreExpressions}{\textsf{\sl MoreExpressions}}
\setspecialsymbol{\Tree}{\textsf{\sl Tree}}
\setspecialsymbol{\Element}{\textsf{\sl Element}}

\setspecialsymbol{\CondClauseList}{\nonterminal{CondClauseList}}
\setspecialsymbol{\CondClause}{\nonterminal{CondClause}}
\setspecialsymbol{\OptElseClause}{\nonterminal{OptElseClause}}

\setspecialsymbol{\Expressiona}{\textsf{\sl Expression$_1$}}
\setspecialsymbol{\Expressionb}{\textsf{\sl Expression$_2$}}
\setspecialsymbol{\Expressionk}{\textsf{\sl Expression$_k$}}
\setspecialsymbol{\ExpressionPA}{\textsf{\sl Expression$_{p1}$}}
\setspecialsymbol{\ExpressionPB}{\textsf{\sl Expression$_{p2}$}}
\setspecialsymbol{\ExpressionPK}{\textsf{\sl Expression$_{pk}$}}

\setspecialsymbol{\ExpressionVA}{\textsf{\sl Expression$_{c1}$}}
\setspecialsymbol{\ExpressionVB}{\textsf{\sl Expression$_{c2}$}}
\setspecialsymbol{\ExpressionVK}{\textsf{\sl Expression$_{ck}$}}
\setspecialsymbol{\ExpressionVE}{\textsf{\sl Expression$_{else}$}}

\setspecialsymbol{\Namea}{\textsf{\sl Name$_1$}}
\setspecialsymbol{\Nameb}{\textsf{\sl Name$_2$}}
\setspecialsymbol{\Namek}{\textsf{\sl Name$_k$}}

\setspecialsymbol{\Bindings}{\nonterminal{Bindings}}
\setspecialsymbol{\NameG}{\nonterminal{Name}}
\setspecialsymbol{\Rest}{\nonterminal{Rest}}
\setspecialsymbol{\ExpressionG}{\nonterminal{Expression}}
\setspecialsymbol{\ExpressionBody}{\textsf{\sl Expression$_{body}$}}
\setspecialsymbol{\ExpressionP}{\textsf{\sl Expression$_{predicate}$}}
\setspecialsymbol{\ExpressionV}{\textsf{\sl Expression$_{consequent}$}}

\setspecialsymbol{\n}{\textsf{\sl n}}
\setspecialsymbol{\fillspace}{\underline{\ \ \ \ \ \ \ \ \ \ }}

\setspecialsymbol{\ak}{$a_k$}
\setspecialsymbol{\bk}{$b_k$}
\setspecialsymbol{\rk}{$r_k$}
\setspecialsymbol{\ck}{$c_k$}

\setspecialsymbol{\az}{$a_0$}
\setspecialsymbol{\bz}{$b_0$}
\setspecialsymbol{\rz}{$r_0$}
\setspecialsymbol{\cz}{$c_0$}

\setspecialsymbol{\aone}{$a_1$}
\setspecialsymbol{\bone}{$b_1$}
\setspecialsymbol{\rone}{$r_1$}
\setspecialsymbol{\cone}{$c_1$}

\setspecialsymbol{\atwo}{$a_2$}
\setspecialsymbol{\btwo}{$b_2$}
\setspecialsymbol{\rtwo}{$r_2$}
\setspecialsymbol{\ctwo}{$c_2$}

\setspecialsymbol{\athree}{$a_3$}
\setspecialsymbol{\bthree}{$b_3$}
\setspecialsymbol{\rthree}{$r_3$}
\setspecialsymbol{\cthree}{$c_3$}

\setspecialsymbol{\ldots}{$\ldots$}
\setspecialsymbol{\cdots}{$\cdots$}

\setspecialsymbol{\quad}{\mbox{\quad}}
\setspecialsymbol{\squad}{\mbox{\ }}


\renewcommand{\lstlistlistingname}{Code Listings} 
\renewcommand{\lstlistingname}{Code Listing} 
\definecolor{gray}{gray}{0.5} 
\definecolor{key}{rgb}{0,0.5,0} 
  
\lstnewenvironment{pythoncode}[1][]{
\lstset{
language=python,
texcl=true,
basicstyle=\setstretch{1}, % \ttfamily
stringstyle=\sffamily\small,
showstringspaces=false,
identifierstyle=\itshape,
  xleftmargin=20pt,
  xrightmargin=20pt, 
%alsoletter={1234567890},
otherkeywords={\ , \}, \{},
keywordstyle=\bfseries,
upquote=true,
emph={access,and,break,class,continue,def,del,elif,else,%
except,exec,finally,for,from,global,if,import,in,is,%
lambda,not,or,pass,print,raise,return,try,while,assert},
emphstyle=\bfseries,
emph={[2]True, False, None},
emphstyle=[2]\sffamily\small,
emphstyle=[4]\bfseries\itshape,
%emph={[3]1, 2, 3, 4, 5, 6, 7, 8, 9, 0},
columns=fullflexible,
morecomment=[s]{"""}{"""},
keepspaces=true,
commentstyle=\color{gray},
literate=%*{:}{{\textcolor{black}:}}{1}%
   %{"}{{"}}1 
}}{}

\lstnewenvironment{pythoncodewide}[1][]{
\lstset{
language=python,
basicstyle=\setstretch{1}, % \ttfamily
stringstyle=\sffamily\small,
showstringspaces=false,
identifierstyle=\itshape,
  xleftmargin=0pt,
  xrightmargin=0pt, 
%alsoletter={1234567890},
otherkeywords={\ , \}, \{},
keywordstyle=\bfseries,
upquote=true,
emph={access,and,break,class,continue,def,del,elif,else,%
except,exec,finally,for,from,global,if,import,in,is,%
lambda,not,or,pass,print,raise,return,try,while,assert},
emphstyle=\bfseries,
emph={[2]True, False, None},
emphstyle=[2]\sffamily\small,
emphstyle=[4]\bfseries\itshape,
%emph={[3]1, 2, 3, 4, 5, 6, 7, 8, 9, 0},
columns=fullflexible,
morecomment=[s]{"""}{"""},
keepspaces=true,
commentstyle=\color{gray}\slshape,
literate=%*{:}{{\textcolor{black}:}}{1}%
   %{"}{{"}}1 
}}{}

\lstnewenvironment{pythoninteractions}[1][]{
\lstset{
language=python,
basicstyle=\setstretch{1}, % \ttfamily
stringstyle=\sffamily\small,
showstringspaces=false,
identifierstyle=\itshape,
  xleftmargin=20pt,
  xrightmargin=20pt, 
%alsoletter={1234567890},
otherkeywords={\ , \}, \{},
keywordstyle=\color{black},
emph={access,and,break,class,continue,def,del,elif,else,%
except,exec,finally,for,from,global,if,import,in,is,%
lambda,not,or,pass,print,raise,return,try,while,assert},
emphstyle=\color{black}\bfseries,
emph={[2]True, False, None},
emphstyle=[2]\sffamily\small,
columns=fullflexible,
emph={[3]from, import, as},
emphstyle=[3]\color{black},
emphstyle=[4]\bfseries\itshape,
upquote=true,
%morestring=[b]{``},
%morestring=[b]{''},
morecomment=[s]{"""}{"""},
keepspaces=true,
commentstyle=\color{gray}\slshape,
emph={[4]1, 2, 3, 4, 5, 6, 7, 8, 9, 0},
emphstyle=[4]\sffamily\small,
escapechar=\^,
literate=%*{:}{{\textcolor{black}:}}{1}%
   {>>>}{{$\gg$}}2
   %{"}{{"}}1 
}}{}

\lstdefinelanguage{smalltalk}{
  morekeywords={true,false,self,super,nil,class,variable,instance,name,names},
  sensitive=true,
  morecomment=[s]{"}{"},
  morestring=[d]',
  style=smalltalkstyle
}
\lstdefinestyle{smalltalkstyle}{
  literate={:=}{{$\gets$}}1{^}{{$\uparrow$}}1
  keywordstyle=\bfseries,
  columns=fullflexible,
  identifierstyle=\itshape,
  xleftmargin=20pt,
  xrightmargin=20pt, 
}

\lstset{
language=python,
basicstyle=\setstretch{1}, % \ttfamily
stringstyle=\sffamily\small,
showstringspaces=false,
identifierstyle=\itshape,
  xleftmargin=20pt,
  xrightmargin=20pt, 
%alsoletter={1234567890},
otherkeywords={\ , \}, \{},
keywordstyle=\color{black},
emph={access,and,break,class,continue,def,del,elif,else,%
except,exec,finally,for,from,global,if,import,in,is,%
lambda,not,or,pass,print,raise,return,try,while,assert},
emphstyle=\color{black}\bfseries,
emph={[2]True, False, None},
emphstyle=[2]\sffamily\small,
columns=fullflexible,
keepspaces=true,
emph={[3]from, import, as},
emphstyle=[3]\color{black},
emphstyle=[4]\bfseries\itshape,
upquote=true,
%morestring=[b]{``},
%morestring=[b]{''},
morecomment=[s]{"""}{"""},
commentstyle=\color{gray}\slshape,
emph={[4]1, 2, 3, 4, 5, 6, 7, 8, 9, 0},
emphstyle=[4]\sffamily\small,
literate=%*{:}{{\textcolor{black}:}}{1}%
   {>>>}{{$\gg$}}2
   %{"}{{"}}1 
}



\pagestyle{fancy}

\fancyhead{}
\fancyhead[RE]{\nouppercase{\bfseries Exercises and Solutions}} 
\fancyhead[LO]{\nouppercase{\bfseries \leftmark --- Exercises and Solutions}} 
\fancyhead[LE,RO]{\thepage}

%\fancyhead[RE]{\iffloatpage{}{\nouppercase{\bfseries \rightmark}}} 
%\fancyhead[LO]{\iffloatpage{}{\nouppercase{\bfseries \leftmark}}} 
%\fancyhead[LE,RO]{\iffloatpage{}{\thepage}}
\fancyfoot{}
\renewcommand{\headrulewidth}{0.4pt}
\renewcommand{\headsep}{25pt}
%\renewcommand{\headrulewidth}{\iffloatpage{0pt}{0.4pt}}
% http://www.ctan.org/tex-archive/macros/latex/contrib/fancyhdr/fancyhdr.pdf

\fancypagestyle{plain}{
  \fancyhf{} 
 % \fancyfoot[C]{\fbox{\footnotesize David Evans, \emph{Computing: Explorations in Language, Logic, and Machines}, \today}}
  \renewcommand{\headrulewidth}{0pt}
  \renewcommand{\footrulewidth}{0pt}
}

\input{exercises-generated.tex}

\end{document}